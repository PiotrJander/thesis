\documentclass{beamer}
\usetheme{ru}
\usepackage{agda}
\usepackage{color}
\usepackage{mathpartir}
\usepackage{catchfilebetweentags}
\setlength{\mathindent}{0cm}
\let\\=\newline
\renewcommand{\mathcal}[1]{#1}
\input{lhs2tex-prelude}
\title{A Scope-and-Type Safe Universe of Syntaxes with Binding, Their Semantics and Proofs}
\author{Guillaume Allais$^1$, Robert Atkey$^2$, James Chapman$^2$, Conor McBride$^2$, James McKinna$^3$}
\institute{$^1$ Radboud University \\ $^2$ University of Strathclyde \\ $^3$ University of Edinburgh}

\begin{document}

\maketitle

\section{Primer: Scope-and-Type Safe Terms}

  \begin{frame}{Well-Scoped Untyped Lambda Calculus}\Large
    \ExecuteMetaData[lc.tex]{lc}
  \end{frame}

  \begin{frame}{Well-Scoped Untyped Lambda Calculus}\Large
    \ExecuteMetaData[lc.tex]{lcnat}
  \end{frame}

  \begin{frame}{Well-Scoped-and-Typed Lambda Calculus}\Large
    \ExecuteMetaData[lc.tex]{lcfull}
  \end{frame}

  \begin{frame}{Combinators for Indexed Sets}
    \begin{minipage}{0.5\textwidth}
      \ExecuteMetaData[var.tex]{scoped}
    \end{minipage}
    \begin{minipage}{0.3\textwidth}
      \ExecuteMetaData[indexed.tex]{constant}
    \end{minipage}

    \ExecuteMetaData[indexed.tex]{arrow}
    \ExecuteMetaData[indexed.tex]{adjust}
    \ExecuteMetaData[indexed.tex]{forall}

  \end{frame}

  \begin{frame}{Well Scoped-and-Typed Variables and Terms}
    \ExecuteMetaData[var.tex]{var}
    \ExecuteMetaData[motivation.tex]{tm}
  \end{frame}

\section{Primer: Scope-and-Type Safe Programs}

  \begin{frame}{Environments of Values}
    \ExecuteMetaData[environment.tex]{env}
  \end{frame}

  \begin{frame}{Renaming}
    \ExecuteMetaData[motivation.tex]{ren}
  \end{frame}

  \begin{frame}{Substitution}
    \ExecuteMetaData[motivation.tex]{sub}
  \end{frame}

  \begin{frame}{Normalisation by Evaluation}
    \ExecuteMetaData[motivation.tex]{nbe}
  \end{frame}

  \begin{frame}{Observing the Common Structure}
    From
    \begin{mathpar}\huge
      \inferrule{\textit{Var} \longrightarrow \textit{Term}}
                {\textit{Term} \longrightarrow \textit{Term}}
    \end{mathpar}
    To
    \begin{mathpar}\huge
      \inferrule{\textit{V} \longrightarrow \textit{C}}
                {\textit{Env}_V \longrightarrow \textit{Sem}_C}
    \end{mathpar}
  \end{frame}

  \begin{frame}{Extracting the essence of Binding}
    \begin{mathpar}\Huge
      \inferrule{\Box (\textit{V}~\sigma ~\longrightarrow~ \textit{C}~\tau)}
                {\textit{C} (\sigma \Rightarrow \tau)}
    \end{mathpar}
  \end{frame}

  \begin{frame}{Thinning}
    \ExecuteMetaData[environment.tex]{thinning}
    \ExecuteMetaData[environment.tex]{box}
    \ExecuteMetaData[environment.tex]{comonad}
    \ExecuteMetaData[environment.tex]{thinnable}
    \ExecuteMetaData[environment.tex]{freeth}
  \end{frame}

  \begin{frame}{Type-and-Scope Safe Semantics}
    \ExecuteMetaData[motivation.tex]{rsem}
    \uncover<2->{\ExecuteMetaData[motivation.tex]{sem}}
  \end{frame}

\section{Primer: Universe of Data Types}

\begin{frame}{Data as the Fixpoint of a Functor}

  {\huge $\textit{List}_A  = \mu{} X. 1 + A \times X$}\\

  {\huge \begin{tabular}{lcl}
    \textit{nil} & $=$ & $\textit{in} (\textit{inj}_1 ())$ \\
    \textit{cons}$(a, as)$ & $=$ & $\textit{in} (\textit{inj}_2 (a, as))$
  \end{tabular}}
\end{frame}

\begin{frame}{Indexed Data as the Fixpoint of a Functor}

  {\huge \begin{tabular}{ll}
      $\textit{Term}  = \mu{} T. \lambda{} n.$ & $\textit{Fin}(n) + T(1+n)$ \\
      & $+ T(n) \times T(n)$
  \end{tabular}}\bigskip

  {\huge \begin{tabular}{lcl}
    \textit{var}$(k)$    & $=$ & $\textit{in} (\textit{inj}_1 (k))$ \\
    \textit{lam}$(b)$    & $=$ & $\textit{in} (\textit{inj}_2(\textit{inj}_1 (b)))$ \\
    \textit{app}$(f, t)$ & $=$ & $\textit{in} (\textit{inj}_2(\textit{inj}_2 (f, t)))$
  \end{tabular}}
\end{frame}

\begin{frame}{Descriptions of Strictly Positive Functors}
  \ExecuteMetaData[Generic/Data.tex]{desc}
  \ExecuteMetaData[Generic/Data.tex]{interp}
\end{frame}

\begin{frame}{Example: The list functor}
  \ExecuteMetaData[Generic/Data.tex]{listD}
\end{frame}

\begin{frame}{Fixpoints and fold}
  \ExecuteMetaData[Generic/Data.tex]{mu}
  \ExecuteMetaData[Generic/Data.tex]{fmap}
  \ExecuteMetaData[Generic/Data.tex]{fold}
\end{frame}

\section{A Universe of Type-and-Scope Safe Syntaxes}

\begin{frame}{Descriptions of Syntaxes}
  \ExecuteMetaData[Generic/Syntax.tex]{desc}
  \ExecuteMetaData[Generic/Syntax.tex]{interp}
\end{frame}


\begin{frame}{Their Fixpoints: Free Relative Monads}
  \ExecuteMetaData[Generic/Syntax.tex]{mu}
  \ExecuteMetaData[Generic/Syntax.tex]{scope}
\end{frame}

\begin{frame}{Examples: Untyped and Simply-Typed LC}
  \ExecuteMetaData[Generic/Examples/UntypedLC.tex]{LCD}
  \ExecuteMetaData[Generic/Examples/STLC.tex]{stlc}
\end{frame}

\section{Generic Scope-and-Type Safe Programs}

\begin{frame}{A Generic Notion of Semantics}
  \ExecuteMetaData[environment.tex]{kripke}
  \ExecuteMetaData[Generic/Semantics.tex]{semantics}
\end{frame}

\begin{frame}{Fundamental Lemma of Semantics}
  \ExecuteMetaData[Generic/Semantics.tex]{comp}
  \ExecuteMetaData[Generic/Semantics.tex]{semtype}
\end{frame}

\begin{frame}{Proof of the Fundamental Lemma}

  \ExecuteMetaData[Generic/Semantics.tex]{semantics}

  \ExecuteMetaData[Generic/Semantics.tex]{sem}
  \ExecuteMetaData[Generic/Semantics.tex]{body}
\end{frame}

\begin{frame}{Generic Renaming}
  \ExecuteMetaData[Generic/Semantics.tex]{renaming}
\end{frame}
\begin{frame}{Generic Substitution}
  \ExecuteMetaData[Generic/Semantics.tex]{substitution}
\end{frame}

\begin{frame}{From Kripke to Scope: Varlike and Reify}
  \ExecuteMetaData[varlike.tex]{varlike}
  \ExecuteMetaData[Generic/Semantics.tex]{reify}
\end{frame}

\section{Examples of Generic Type-and-Scope Safe Programs}

\begin{frame}{Let-elaboration}
  \ExecuteMetaData[Generic/Examples/ElaborationLet.tex]{letcode}
  \ExecuteMetaData[Generic/Examples/ElaborationLet.tex]{unletcode}
  \ExecuteMetaData[Generic/Examples/ElaborationLet.tex]{unlet}
\end{frame}

\begin{frame}{(Unsafe) Normalisation by Evaluation}
  \ExecuteMetaData[Generic/Examples/NbyE.tex]{domain}
  \hspace{-50pt}\begin{minipage}{0.9\textwidth}
    \ExecuteMetaData[Generic/Examples/NbyE.tex]{nbe-setup}
  \end{minipage}
\end{frame}

\begin{frame}{Example: NBE for Untyped LC}
  Go check src/Generic/Examples/NbyE.agda
\end{frame}

\begin{frame}{Studying a particular syntax}
  \begin{itemize}
    \item Elaboration STLC+State to STLC
    \item An Algebraic Approach to Typechecking (Atkey) (about 12 lines)
    \item POPLMark Challenge Reloaded (SN of STLC via LR) (about 200 lines)
    \item Well Kinded-and-Scoped System F
  \end{itemize}
\end{frame}

\section{Non-Standard Semantics: Binding as Self-Reference}

\begin{frame}{Representing Cyclic Structures}
  \ExecuteMetaData[Generic/Examples/Colist.tex]{clistD}
  \ExecuteMetaData[Generic/Examples/Colist.tex]{clistpat}
  \ExecuteMetaData[Generic/Examples/Colist.tex]{zeroones}
\end{frame}

\begin{frame}{Unrolling the cycles}
  \ExecuteMetaData[Generic/Cofinite.tex]{plug}
  \ExecuteMetaData[Generic/Cofinite.tex]{unroll}
\end{frame}

\begin{frame}{Unfolding the whole structure}
  \ExecuteMetaData[Generic/Cofinite.tex]{cotm}
  \ExecuteMetaData[Generic/Cofinite.tex]{unfold}
\end{frame}

\section{Building Generic Proofs about Generic Programs}

\begin{frame}{Zip: Structurally Equal Values}
 \hspace{-50pt}\begin{minipage}{0.9\textwidth}
   \ExecuteMetaData[Generic/Zip.tex]{ziptype}
 \end{minipage}
\end{frame}

\begin{frame}{A Generic Notion of Simulation}
  Go have a look at src/Generic/Simulation.agda
\end{frame}

\begin{frame}{Fundamental Lemma of Simulations and Corollary}
 \hspace{-50pt}\begin{minipage}{0.9\textwidth}
   \ExecuteMetaData[Generic/Simulation.tex]{simbody}
   \ExecuteMetaData[Generic/Simulation.tex]{rensub}
 \end{minipage}
\end{frame}

\begin{frame}{Fundamental Lemma of Fusion}
  Go have a look at src/Generic/Fusion.agda
\end{frame}

\section{Conclusion}

\begin{frame}
  \begin{itemize}
    \item Well Scoped-and-Typed STLC
    \item Well Scoped-and-Typed Programs on STLC
    \item Well Scoped-and-Typed Generic Semantics on STLC
    \item Universe of Datatypes and Generic Fold
    \item Universe of Syntaxes and Generic Semantics
    \item Renaming, Substitution, Normalisation by Evaluation
    \item But also Printing, Let-elaboration, ...
    \item Non-Standard Semantics: Binding as Self-Reference
    \item Generic Proofs too!
  \end{itemize}
\end{frame}

\end{document}
