\documentclass[bsc,frontabs,oneside,singlespacing,parskip,deptreport]{infthesis}

\batchmode
\usepackage{agda}
\usepackage{catchfilebetweentags}
\usepackage{amsthm}
\usepackage{amssymb}
\usepackage{amsmath}
\usepackage{url}
\usepackage{enumitem}
\usepackage{tikz-cd}
\usepackage[hidelinks]{hyperref}

%%%%%%%%%% AGDA ALIASES

\newcommand{\APT}{\AgdaPrimitiveType}
\newcommand{\AK}{\AgdaKeyword}
\newcommand{\AM}{\AgdaModule}
\newcommand{\AS}{\AgdaSymbol}
\newcommand{\AStr}{\AgdaString}
\newcommand{\AN}{\AgdaNumber}
\newcommand{\AD}{\AgdaDatatype}
\newcommand{\AF}{\AgdaFunction}
\newcommand{\AR}{\AgdaRecord}
\newcommand{\ARF}{\AgdaField}
\newcommand{\AB}{\AgdaBound}
\newcommand{\AIC}{\AgdaInductiveConstructor }
\newcommand{\ti}{\textasciitilde}
\newcommand{\tis}{\textasciitilde\,\,}
\newcommand{\lcl}{λ\textsuperscript{cl}}
\newcommand{\stlc}{λ\textsuperscript{$\rightarrow$}}

\newcommand{\Set}{\mathbf{Set}}


\theoremstyle{definition}
\newtheorem*{definition}{Definition}

\theoremstyle{lemma}
\newtheorem*{lemma}{Lemma}


\setcounter{secnumdepth}{3}

\begin{document}

\title{Verifying type- and scope-safe program transformations}

\author{Piotr Jander}

\course{Master of Informatics}
\project{{\bf MInf Project (Part 2) Report}}

\date{\today}

\abstract{ There is an ongoing effort in the programming languages
  community to verify correctness of compilers. Type- and scope-safe
  representation is a commonly used encoding for intermediate
  languages, which however requires writing considerable
  meta-theoretical boilerplate for each IR. Standard techniques for
  showing correctness of compiler transformations are bisimulations
  and Kripke logical relations.

  This project formalises a language with closures, implements a
  closure conversion algorithm, and mechanises the proof of its
  correctness using the the aforementioned techniques.

  This project also suggest that a language with closures cannot
  benefit from state-of-the-art techniques for reducing
  meta-theoretical boilerplate with generic programming.
}

\maketitle

\section*{Acknowledgements}
Acknowledgements go here.

\tableofcontents

%\pagenumbering{arabic}


\chapter{Introduction}

TODO

\chapter{Related literature}

\section{Verified compilation}

Closure conversion is just one possible verification phase, and its
verification constitutes part of a wider effort to verify compilation
end-to-end, which usually entails verifying operational correctness of
all compilation phases.

As far as type safety is concerned, the reference is a paper by
Morrisett et al., "From System F to Typed Assembly Language"
\cite{TODO}. It builds upon previous results in type safety of
compilation phases (like the aforementioned \cite{TCC}) and describes
a typed RISC-like assembly (named TAL), which is the target of the
final phases of compilation. As a whole, the paper proves type safety
for a compilation pipeline from System F to TAL. It does not, however,
prove end-to-end operational correctness.

An compiler which was verified for end-to-end operational
correctness was described by Adam Chlipala in his paper "A Certified
Type-Preserving Compiler from Lambda Calculus to Assembly
Language". The source is a variant of the simply-typed lambda calculus
(STLC). Compilation proceeds through six phases, eventually yielding
idealised assembly code. The compiler is implemented in Coq, where
terms and functions on terms are dependently typed, guaranteeing type
preservation. This is also the approach taken in this project, except
that we use Agda instead of Coq \cite{TODO}. Operational
correctness is proved by adopting denotational semantics, unlike in
this project, which uses operational semantics. Due to unfamiliarity
with operational semantics, we cannot comment on which approach is
better (TODO or can we?).

Another example of a certified compiler is CompCert
\cite{TODOcompcert}, which is the result of the first successful
attempt to implement a certified compiler of a real-world (TODO
wording) language. Even compared with the simply-typed lambda calculus
(STLC), which was the source language in Chlipala's work \cite{TODO},
the C language is in some ways simpler, especially since it does not
have first-class functions with free variables (TODO wording:
scoping?). But, being a fully-fledged language, C presents enough
challenges as the source language of a verified compiler.

\section{Closure conversion}

Closure conversion is a compilation phase where functions or lambda
abstractions with free variables are transformed to /closures/. A
closure consists of a body (code) and the /environment/, which is a
record holding the values corresponding to the free variables in the
body (code). Closure conversion transforms abstractions to closures,
and replaces references to variables with lookups in the environment.

Closure conversion was necessarily used in every compiler for a
language which supports functions with free variables (TODO wording:
scope?). But the first work which provided a rigorous treatment of
closure conversion was the paper "Typed Closure Conversion" by
Minamide et al. \cite{TODO}. It demonstrated type-preserving closure
conversion, where closure environments have existential types (TODO
wording). On top of a proof of type-safety, the paper contains a proof
of operational correctness of the typed closure conversion algorithm
by logical relations.

Another notable paper about closure conversion is "Typed Closure
Conversion Preserves Observational Equivalence" by Ahmed and Blum
\cite{TODO}. The paper's title explains its main result, so we should
explain the title.

(TODO bring up the Reynolds' paper) Within a language L, we have a
program P = C[A], where A is an implementation of an abstraction and C
is the "context", or "the rest of the program". Given some other
implementation A' of the abstraction, we say that A and A' are
contextually equivalent when for all possible contexts C, programs P =
C[A] and P' = C[A'] behave identically.

We say that another abstraction A' is contextually equivalent to A if
for all contexts C, programs C[A] and C[A'] are equivalent. This
corresponds to a programmer's intuition that A and A' behave in the
same way in all possible programs.

TODO OE matters for security and safety: If an attack would be
possible by exposing a certain implementation detail, then this detail
is made inaccessible / private, for example by using an existential
type.

Why this matters: modern software systems are made up of multiple
components, of which some might not be trusted.

// To ensure reliable and secure operation, it is important to defend
against faulty or malicious code. Language-based security is built
upon the concept of abstraction: if access to some private
implementation detail might enable an attack, then this detail is made
inaccessible by hiding it behind an abstract interface, for example
using an existential type. //

TODO I have quite a bit about the paper and we don't want to duplicate
the paper's introduction: how do I make it shorter?

\chapter{Background}

This chapter will introduce the relevant concepts. It will start with
closure conversion, then discuss compilation phases and intermediate
languages, and finally explain the Agda definitions and encodings
which were borrowed from ACMM and PLFA.

\section{Closure conversion}

TODO explain and give an example

TODO explain why existential types

\section{Compilation phases and intermediate representations}

In all but the most trivial compilers, compilation proceeds in
phases, or transformations. A compilation phase transforms the
compilation unit to bring it one step closer from the source code to
the target representation.

[diagram here]

\paragraph{Intermediate representations} As illustrated in the figure,
each compilation phase takes a source representation to a target
representation [relate to diagram]. An intermediate representation can
also be called an intermediate language, and abbreviated to IR or
IL. For some phases, the source and target representation may be the
same. Arguably, this is the case for constant expression folding.

However, other phases benefit from using different source and target
representations. An example of such transformation is closure
conversion, which as the reader may recall from [section], transforms
abstractions with free variables to so called closures, which take an
explit environement and can only reference values from that
environment.

\paragraph{Typed and untyped IRs} To question of whether closure
conversion must necessarily use different source and target languages
hinges on the distinction between typed and untyped intermediate
language. Using a typed IR requires that at each point along the
compilation pipeline, intermediate representantions are well-typed.

Suppose that closure conversion is performed on simply typed lambda
calculus (STLC). One of the two in necessary for the target language
of closure conversion: either it should have first-class closures, or
existential types. Neither is true of STLC, so another intermediate
language is needed.

[TODO unintellegible comment about this paragraph] On the other hand,
if the source and target representations are untyped, then the
compiler architect might get away with using the same intermediate
language as both source and target (for example Scheme, which is
sometimes used as a compilation target). But even in this case,
compilation process might benefit if the abstract syntax has explicit
closures.

\paragraph{IRs in this project} This project uses a dependently typed
meta language (Agda) to implement compilation phases (specifically,
closure conversion), so typed intermediate representations are a
natural choice. Therefore, in the following sections, we will describe
two intermediate representations, which are both variants of lambda
calculus. The source representation will be simply typed lambda
calculus, which we will refer to as λst. The target
representation will be simply typed lambda calculus with closures,
denoted with λcl.

The two intermediate representations are similar, and differ mainly in
having either abstractions with free variables in λst or closures with
environments in λcl. Unfortunately, this means that formalisations of
λst and λcl share a lot of duplication. This is a common problem in
formalising languages which has recently been addressed by
\cite{DBLP:journals/pacmpl/AllaisA0MM18}. Whether techniques from
Allais et al. are applicable to this work will be discussed in [related
work]. On the other hand, [section] demonstrates that while two
intermediate languages can only differ in a handful of syntactic
constructs and reduction steps, they can behave very differently with
respect to the ubiquitious operations of renaming and substitution.

\section{Type- and scope-safe representation of simply typed lambda
  calculus λst}

This section will discuss the encoding of simply typed lambda calculus
(abbreviated as STLC, denoted with λst), which is the source language
of closure conversion. Typing and reduction rules are standard for
call-by-value lambda calculus, so it is the encoding in Agda which is
of interest in this section. As similar encoding is used for the
closure language λcl.

Using dependently typed Agda as the meta language allows us to encode
certain invariants in the representation. Two such invariants are
scope and type safety. The representation is scope-safe in the sense
that all variables in a term are either bound by some binder in the
term, or explicitly accounted for in the context. It is type-safe in
the sense that terms are synonymous with their typing derivations,
which makes ill-typed terms unrepresentable. This kind of scope and
type safety is due to \cite{DBLP:conf/csl/AltenkirchR99}. The rest of
this section shows how this is achieved in Agda; the Agda encoding is
based on the one used in \cite{DBLP:conf/cpp/Allais0MM17},
\cite{DBLP:journals/pacmpl/AllaisA0MM18}, and
\cite{DBLP:conf/sbmf/Wadler18}.

TODO STLC as a figure here

To start with, λst typed are defined as follows.

\ExecuteMetaData[StateOfTheArt/Types.tex]{type}

The context is simply a list of types.

\ExecuteMetaData[StateOfTheArt/Types.tex]{context}

Variables are synonymous with proofs of context membership. Since a
variable is identified by its position in the context, it is
appropriate to call it a de~Bruijn variable. Accordingly, the
constructors of \AS{Var} are named after \textit{zero} and
\textit{successor}. Notice that the definition assumes that the
leftmost type in the context corresponds to the most recently bound
variable.

\ExecuteMetaData[StateOfTheArt/Types.tex]{var}

We can now present the formulation of λst terms, which is synonymous
with their typing derivations:

\ExecuteMetaData[StateOfTheArt/STLC.tex]{terms}

The syntactic variable \AS{V} constructor takes a de~Bruijn variable to
a term. The abstraction constructor \AS{L} requires that the body is
well-typed in the context \AS{Γ} extended with the type \AS{σ} of the
variable bound by the abstraction. The application constructor
\AS{A} follows the typing rule for application.


\section{Type- and scope-safe programs}
\label{sec:typ-scop-saf-prog}

Many useful traversals of the abstract syntax tree involve maintaining
a mapping from free variables to appropriate values. Two such
traversals are simultaneous renaming and substitution.

Simultaneous renaming takes a term \AS{N} in the context \AS{Γ}. It maintains
a mapping \AS{ρ} from variables in the original context \AS{Γ} to
\textit{variables} in some other context \AS{Δ}. It produces a term in
\AS{Δ}, which is \AS{N} with variables renamed with \AS{ρ}.

Similarly, simultaneous substitution takes a term \AS{N} in the context
\AS{Γ}. It maintains a mapping \AS{σ} from variables in the original context
\AS{Γ} to \textit{terms} in some other context \AS{Δ}. It produces a
term in \AS{Δ}, which is \AS{N} with variables substitution for with \AS{σ}.

Before we can demonstrate an implementation of renaming and
substitution, we need to formalise the notion of a mapping from free
variables to appropriate values, which we call the
\textit{environment}.

\ExecuteMetaData[STLC.tex]{env}

A environment \AS{(Γ ─Env) 𝓥 Δ} encapsulates a mapping from variables in
\AS{Γ} to values \AS{𝓥} (variables for renaming, terms for
substitution) which are well-typed and -scoped in \AS{Δ}.

An environment which maps variables to variables is important enough
to deserve its own name.

\ExecuteMetaData[STLC.tex]{thinning}

There is a notion of an empty environment \AS{ε}, of extending an
environment \AS{ρ} with a value \AS{v}: \AS{ρ ∙ v}, and of mapping a
function \AS{f} over an environment \AS{ρ}: \AS{f <\$> ρ},
corresponding to the analogous operations on contexts (which are just
lists). Finally, \AS{select ren ρ} renames a variable with \AS{ren}
before looking it up in \AS{ρ}.

\ExecuteMetaData[STLC.tex]{envops}

Notice that those four operations on environments are defined using
copatterns \cite{DBLP:conf/popl/AbelPTS13} by `observing` the
behaviour of \AS{lookup}.

Equipped with the notion of environments, we can give an
implementation of renaming and substitution:

\ExecuteMetaData[StateOfTheArt/STLC.tex]{rename}
\ExecuteMetaData[StateOfTheArt/STLC.tex]{subst}

Notice that those two traversals are indentical except (1)
\textit{renaming} wraps the result of \AS{lookup ρ x} in \AS{V}, and
\textit{renaming} and \textit{substitution} extend the environment in
a different way: \AS{s <\$> ρ ∙ z} vs \AS{rename (pack s) <\$> σ ∙ V
  z}. The observation that renaming and substitution for STLC share a
common structure was a basis was the unpublished manuscript by McBride
\cite{mcbride2005type}, and subsequently motivated the ACMM paper
\cite{DBLP:conf/cpp/Allais0MM17}. In [section], we will show how ACMM
abstracts this common structure of renaming and substitution into a
notion of a semantics.

Also notice how the functions \AS{ext} and \AS{exts} extend the
environment when the traversal goes under a binder.

An example instantation of simultaneous substitution is single
substitution. Single substitution replaces occurrences of the
last-bound variable in the context, and it is useful for defining the
beta reduction for abstractions. Single substitution environment is an
identity substitution environment extended with a single value:

\ExecuteMetaData[StateOfTheArt/STLC.tex]{id-subst}

\ExecuteMetaData[StateOfTheArt/STLC.tex]{single-subst}

\section{ACMM's notion of a semantics}

TODO ACMM, synch, fusions

\section{Small-step operational semantics}

The formalisation of small step semantics for a call-by-value lambda
calculus is adapted from \cite{DBLP:conf/sbmf/Wadler18}.

Values are terms which do not reduce further. In this most basic
version of lambda calculus language, the only values are abstractions:

\ExecuteMetaData[StateOfTheArt/STLC.tex]{values}

Our operational semantics include two kinds of reduction rules. Compatibility rules, whose
names start with \AS{ξ}, reduce parts of the term (specifically, the LHS
and RHS of application). Beta reduction \AS{β-L}, on the other hand,
describes what an abstraction applied to a value reduces to.

\ExecuteMetaData[StateOfTheArt/STLC.tex]{reductions}

A term which can take a reduction step is called a reducible
expression, or a redex. A property of a language that every well-typed
term is either a value or a redux is called type-safety. This property
is captured by a slogan `well-typed terms don't get stuck` and can be
proved by techniques like `progress and preservation` or logical
relations. Simply typed lambda calculus is type-safe, and so is this
formalisation. For a proof of type safety for a similar formalisation
of STLC, cf. \cite{DBLP:conf/sbmf/Wadler18}.

Operational semantics are needed for the treatment of bisimulation.

\chapter{Formalising closure conversion}
\label{cha:agda-development}

This chapter presents this project's formalisation of closure
conversion. It starts by discussing the closure language λcl, an
intermediate language which is like STLC but with abstractions
replaced by closures. Then it demonstrates a type-preserving
conversion for λst to λcl which has the property that the obtained
closure environments are `minimal`.  Finally, several properties about
interactions between renaming and substitution in λcl are formally
established --- they are needed in proofs of correctness in subsequent
chapters.

\section{Closure language λcl}
\label{sec:closure-language-cl}

As discussed in the Background [or maybe Intro?] chapter, some
compilation phases must use different source and target intermediate
representations. This is the case with closure conversion, and this
section presents a formalisation of an intermediate language with
closures. The language is very similar the formalised simply typed
lambda calculus, except that abstraction with free variables are
replaced by closures with environments. What might seem like a simple
change has interesting implications for traversals like renaming and
substitution.

The closure language λcl shares types, contexts, and
de-Bruijn-variables-as-proofs-of-context-membership, and their
respective Agda formalisations, with the source representation. In
general, two different intermediate representations do not need to
share the same type system, but if they do, this simplifies
formalisation. The descriptions of those formalisations can be found
in Section~[TODO].

\subsection{Terms}
\label{sec:closure-language-cl-1}

The definition of terms of λcl differs from terms of λst in the \AS{L}
constructor, which, in λcl, holds the closure body and the closure
environment.

\ExecuteMetaData[StateOfTheArt/Closure.tex]{terms}

Notice that the typing rule for the closure constructor \AS{L}
mentions two contexts, \AS{Γ} and \AS{Δ}. We call \AS{Γ} the
\textit{outer context} and \AS{Δ} the \textit{inner context} of a
closure.

\begin{minipage}{.5\textwidth}
  \[
  \frac
  {\Gamma , x : \sigma \vdash e : \tau}
  {\Gamma \vdash \lambda x : \sigma . e : \sigma \rightarrow \tau}
  \text{T-abs}
  \]
\end{minipage}%
\begin{minipage}{.5\textwidth}
  \[
  \frac
  {e_{ev} = subst ( \Delta \subseteq \Gamma ) \quad \quad \Delta , x : \sigma \vdash e : \tau}
  {\Gamma \vdash \langle\langle \lambda x : \sigma . e \; , \; e_{ev} \rangle\rangle : \sigma \rightarrow \tau}
  \text{T-clos}
  \]
\end{minipage}

The closure as a whole is typed in \AS{Γ}, but the closure body (also
called the \textit{closure code}) is typed in \AS{σ ∷ Δ}. The
relationship between \AS{Γ} and \AS{Δ} is given by the closure
environment.

A closure environment is traditionally implemented as a record, and
variables in the closure code reference fields of that record. In this
development, on the other hand, the environment is represented as a
substitution environment, that is, a mapping from variables in \AS{Δ}
to terms in \AS{Γ}. This representation is isomorphic to the one using
a record, and it has several benefits, especially eliminating the need
for products in the language, and overall simplification of the
formalisation.

Finally, recall from [section] that in order for a closure-converted
program to be well-typed, a closure environment should have an
existential type. It is important to note that in this formalisation,
existential typing is achieved in the meta language Agda, not in the
object language λcl, which does not have existential types. Indeed,
existential quantification (including over types) can in achieved in
Agda through dependent products, a datatype constructor is a dependent
product, and the environment is a parameter to the \AS{L} constructor.

\subsection{Renaming and substitution}
\label{sec:renam-subst}

Consider the case for the constructor \AS{L} of renaming and
substitution in λcl and how it is different from the corresponding
definition in λst.

\ExecuteMetaData[StateOfTheArt/Closure.tex]{rename}
\ExecuteMetaData[StateOfTheArt/Closure.tex]{subst}

Unlike in λst, renaming and substitution in λcl \textit{do not go
  under binders} (do not change the closure body). This is because
renaming and substitution take a term in a context \AS{Γ} to a term in
a context \AS{Γ'}. But the code (body) of a closure is typed in a
different context \AS{Δ}. So upon recursing on a closure, renaming and
substitution adjust the closure environment and leave the closure body
unchanged. The adjustment to the environment is \AS{rename ρ <\$> E}
in the case of renaming and \AS{subst ρ <\$> E} in the case of
substitution. In either case, the adjustment consists of mapping the
renaming/substitution over the values in the environment.

The fact that in λcl, renaming and substitution do not go under
binders will allow us to prove `fusion lemmas` in [section] without
using the machinery of ACMM, which will significantly simplify the
proofs.

Just like in λst, we also define functions \AS{ext} and \AS{exts}
which extend the environment when renaming or substitution goes under
a binder:

\ExecuteMetaData[StateOfTheArt/Closure.tex]{ext}
\ExecuteMetaData[StateOfTheArt/Closure.tex]{exts}

\subsection{Operational semantics}
\label{sec:oper-semant}

Operational semantics are similar to the semantics for λst, except for
adjustments for closures. Values in λcl are closures, and the rule for
beta reduction is different:

\ExecuteMetaData[StateOfTheArt/Closure.tex]{beta}

Recall that a closure is a function without free variables,
partially applied to an environment. When the closure argument reduces
to a value, the argument and the values in the environment get
simultaneously substituted into the closure body. The simplicity of
this reduction rule is another benefit of representing environments as
substitution environments.

\subsection{Conversion from λst to λcl}
\label{sec:conversion-from-st}

This project's approach to typed, or type-preserving, closure
conversion follows \cite{DBLP:conf/popl/MinamideMH96}. An important
point here is that the specification of typed closure conversion
allows for different implementations which might differ in their
treatment of environments. The only requirement in the specification
is that

\begin{enumerate}
\item If the source term is an abstraction typed in the context
  \AS{Γ};
\item if the body of the source abstraction can be typed in a smaller
  context \AS{Δ}, such that \AS{Δ ⊆ Γ};
\item then the target terms is a closure whose environment is a
  substitution from \AS{Δ} to \AS{Γ}.
\end{enumerate}

This is given by the following conversion rule:

\[
  \frac
  {e_{ev} = subst (\Delta \subseteq \Gamma) \quad \quad \Delta , x : \sigma \vdash e \leadsto e' : \tau }
  {\Gamma \vdash \lambda x : \sigma . e \leadsto
    \langle\langle \lambda x : \sigma . e' \; , \; e_{ev} \rangle\rangle : \sigma \rightarrow \tau}
\]

It is up to the implementation of closure conversion to decide how big
to make \AS{Δ}, on the spectrum between (1) \AS{Δ} being equal to
\AS{Γ}, and (2) \AS{Δ} being `minimal`, i.e. only containing the parts
of \AS{Γ} which are necessary to type the term. We present two Agda
implementation of closure conversion, corresponding to the two ends of
the spectrum.

Closure conversion where \AS{Δ} is the same as \AS{Γ} is a simple
transformation:

\ExecuteMetaData[StateOfTheArt/Bisimulation.tex]{convert}

where \AS{T.id-subst} is the identity substitution which maps a term
in \AS{Γ} to itself, defined as:

\ExecuteMetaData[StateOfTheArt/Closure.tex]{id-subst}

We call the other end of the spectrum \textit{minimising closure
  conversion}. Its implementation in Agda is rather more involved and
is described in the next section.

\subsection{Minimising closure conversion}
\label{sec:minim-clos-conv}

Minimising closure conversion is given by the following deduction
rules, where a statement \AS{Γ ⊢ e : σ ↝ Δ ⊢ e' : σ} should be read as:
`the term \AS{e} of type \AS{σ} in the context \AS{Γ} can be closure
converted to the term \AS{e'} in \AS{Δ}`:

\begin{minipage}{.5\textwidth}
  \[
    \frac
    {}
    {\Gamma \vdash x : \sigma \leadsto \emptyset , x : \sigma \vdash x : \sigma}
    \;\text{(min-V)}
  \]
\end{minipage}%
\begin{minipage}{.5\textwidth}
  \[
    \frac
    {
      \begin{matrix}
        \Gamma \vdash e_1 : \sigma \to \tau \leadsto \Delta_1 \vdash
        e_1' : \sigma \to \tau \\
        \Gamma \vdash e_2 : \sigma \leadsto \Delta_2 \vdash e_2' :
        \sigma \\
        \Delta = merge \; \Delta_1 \; \Delta_2
      \end{matrix}
      }
    {\Gamma ⊢ e_1 e_2 : \tau  \leadsto \Delta  \vdash e_1' e_2' : \tau}
     \;\text{(min-A)}
  \]
\end{minipage}

\[
  \frac {\Gamma , x : \sigma ⊢ e : \tau \leadsto \Delta , x : \tau
    \vdash e : \tau \quad \quad e_{id} = subst ( \Delta \subseteq
    \Delta )}
  {\Gamma \vdash \lambda x : \sigma . e : \sigma \to \tau \leadsto
    \Delta \vdash \langle\langle \lambda x : \sigma . e \; , \; e_{id}
    \rangle\rangle : \sigma \to \tau} \; \text{(min-L)}
\]

\textbf{min-V}: Any variables can be typed in a singleton context
containing just the type of the variable.

\textbf{min-A}: If the conversion \AS{e₁'} of \AS{e₁} can be typed in
$\Delta_1$, and the conversion \AS{e₂'} of \AS{e₂} can be typed in \AS{Δ₂},
then the application \AS{e₁' e₂'} can be typed in \AS{Δ}, where \AS{Δ}
is the result of merging \AS{Δ₁} and \AS{Δ₂}.

\textbf{min-L}: If the conversion \AS{e'} of the abstraction body
\AS{e} can be typed in context \AS{σ ∷ Δ} (or $\Delta, x : \sigma$,
using the notation with names), then the closure resulting from the
conversion of the abstraction can be typed in \AS{Δ}, and it has the
identity environment $\Delta \subseteq \Delta$.

To formalise this conversion in Agda, we need several helper
definitions.

\subsubsection{Merging subcontexts}
\label{sec:merging-subcontexts}

The deduction rules for minimising closure conversion contained
statements of the form \AS{Δ ⊆ Γ}, which reads: `\AS{Δ} is a
subcontext of \AS{Γ}`. Since in this development, a context is just a
list of types, the notion of subcontexts can be captured with the
\AS{\_⊆\_} (sublist) relation from Agda's standard library. The
inductive definition of the relation is:

\ExecuteMetaData[StateOfTheArt/Sublist.tex]{sublist}

This project's contribution is to define the operation of merging two
subcontexts. Given contexts \AS{Γ}, \AS{Δ}, and \AS{Δ₁} such that
\AS{Δ ⊆ Γ} and \AS{Δ₁ ⊆ Γ}, the result of merging the subcontexts
\AS{Δ} and \AS{Δ₁} is a context \AS{Γ₁} which satisfies the following
conditions:

\begin{enumerate}
\item It is contained in the big context: \AS{Γ₁ ⊆ Γ}.
\item It contains the small contexts: \AS{Δ ⊆ Γ₁} and \AS{Δ₁ ⊆ Γ₁}.
\item The proof that \AS{Δ ⊆ Γ} obtained by transitivity from \AS{Δ ⊆
    Γ₁} and \AS{Γ₁ ⊆ Γ} is the same as the input proof that \AS{Δ ⊆
    Γ}; similarly for \AS{Δ₁ ⊆ Γ}.
\end{enumerate}

All those requirements are captured by the following dependent record
in Agda:

\ExecuteMetaData[StateOfTheArt/SubContext.tex]{sublistsum}

The type of the function which merges two subcontexts can be stated
as:

\ExecuteMetaData[StateOfTheArt/SubContext.tex]{merge}

We argue that the type of the function completely captures its
behaviour (TODO how would we prove this?). The fact that a type can
completely capture the behaviour of a function is a remarkable feature
of programming with dependent types. Even more remarkable is the fact
that the logical properties of \AS{Γ₁} are useful computationally. E.g
the proof that \AS{Δ ⊆ Γ₁} determines a renaming from \AS{Δ} to
\AS{Γ₁}, which is used in the minimising closure conversion
algorithm. A further example: the fact that \AS{⊆-trans Δ⊆Γ₁ Γ₁⊆Γ ≡
  Δ⊆Γ} is used in proofs of certain equivalences involving subcontexts
and renaming.

\subsection{Agda implementation of minimising closure conversion}
\label{sec:agda-impl-minim}

Recall that terms of our intermediate languages are explicitly typed
in a given context. For that reason, the result type of minimising
closure conversion must be existentially quatified over a
context. In fact, the context should be a subcontext of the input
context \AS{Γ}. This is captured with the dependent record
\AS{\_⊩\_}:

\ExecuteMetaData[StateOfTheArt/ClosureConversion.tex]{ex-subctx-trm}

For example, a term \AS{N} in a context \AS{Δ} which is a subcontext
of \AS{Γ} by \AS{Δ⊆Γ}, would be constructed as \AS{∃[ Δ ] Δ⊆Γ ∧ N}.

With this data type, the type of the minimising closure conversion
function is:

\ExecuteMetaData[StateOfTheArt/ClosureConversion.tex]{min-cc}

The function definition is by cases:

\textbf{Variable case}

\ExecuteMetaData[StateOfTheArt/ClosureConversion.tex]{min-cc-v}

Following \textit{min-V}, a variable is typed in a singleton
context. The proof of the subcontext relation is computed from the
proof of the context membership by a function \AS{Var→⊆}.

\textbf{Application case}

\ExecuteMetaData[StateOfTheArt/ClosureConversion.tex]{min-cc-a}

Given an application \AS{e₁ e₂}, \AS{e₁} and \AS{e₂} are closure
converted recursively, resulting in terms \AS{e₁'} and \AS{e₂'}, which
are typed in \AS{Δ₁} and \AS{Δ₂}, respectively. Following
\textit{app-V}, the result of closure-converting the application is
typed in the context \AS{Δ}, which is the result of merging \AS{Δ₁}
and \AS{Δ₂}. As terms are explicitly typed in a context, \AS{e₁'} and
\AS{e₂'} have to be renamed from \AS{Δ₁} to \AS{Δ}, and from \AS{Δ₂}
to \AS{Δ}, respectively. A renaming environment is computed from a
subcontext relation proof by the function \AS{⊆→ρ} which is given by:

\ExecuteMetaData[StateOfTheArt/ClosureConversion.tex]{subctx-to-ren}

\textbf{Abstraction case}

\ExecuteMetaData[StateOfTheArt/ClosureConversion.tex]{min-cc-l}

Following \textit{min-A}, the result of closure-converting an
abstraction depends on the result \AS{N†} of closure-clonverting its
body. A recursive call on the body of the abstraction yields a term
typed in some context \AS{Δ}. But looking at the typing rule for
closures (\textit{T-clos}), the closure body is typed in a context
\AS{σ ∷ Δ₁} (or \AS{Δ₁, x : σ} using named variables), where \AS{σ} is
the type of the last bound variable and \AS{Δ₁} is the context
corresponding to the closure environment. Thus, we need a way of
decomposing \AS{Δ} into \AS{σ} and \AS{Δ₁}, together with an
appropriate proof of membership in the input context \AS{Γ}.

This task is achieved by the function \AS{adjust-context}:

\ExecuteMetaData[StateOfTheArt/ClosureConversion.tex]{adjust-context-f}

whose specification is captured by its return type which uses the
dependent record \AS{AdjustContext}:

\ExecuteMetaData[StateOfTheArt/ClosureConversion.tex]{adjust-context-t}

The specification is: given \AS{Δ ⊆ A ∷ Γ}, there exists a context
\AS{Δ₁} such that \AS{Δ₁ ⊆ Γ} and \AS{Δ ⊆ A ∷ Δ₁}, such that the
proof \AS{Δ ⊆ A ∷ Γ} obtained by transitivity is the same as the input
proof.

The evidence that \AS{Δ ⊆ A ∷ Δ₁} is used to rename \AS{N†} so that
the final inherently-typed term is well-typed.

***

We also provide a wrapper function \AS{\_†}:

\ExecuteMetaData[StateOfTheArt/ClosureConversion.tex]{dag}

This function is a wrapper over the \AS{min-cc} function which undoes
the minimisation on the outer level. In other words, all closures in
the term are still minimised, but the outer term is typed in the same
context as the input source term. This is useful when we need to
compare the input and output of closure conversion, and need to ensure
that they are typed in the same context.

\subsection{Fusion lemmas for the closure language λcl}

When studying the meta-theory of a calculus, one systematically needs
to prove fusion lemmas for various traversals. A fusion lemma relates
three traversals: the pair we sequence and their sequential
composition. The two traversals which have to be fused in later proofs
are renaming and substitution. There are four ways we can sequence
renaming and substitution, and each of those four sequencing can be
expressed as a single renaming or substitution:

\begin{enumerate}[nolistsep]
  \item A renaming followed by a renaming,
  \item A renaming followed by a substitution,
  \item A substitution followed by a renaming,
  \item A substitution followed by a substitution.
\end{enumerate}

We state the results as signatures of Agda functions, using the
environment combinators \AS{\_<\$>\_} and \AS{select} which are described
in Section~\ref{sec:typ-scop-saf-prog}.

\ExecuteMetaData[StateOfTheArt/Closure-Thms.tex]{rename-rename}
\ExecuteMetaData[StateOfTheArt/Closure-Thms.tex]{subst-rename}  
\ExecuteMetaData[StateOfTheArt/Closure-Thms.tex]{rename-subst} 
\ExecuteMetaData[StateOfTheArt/Closure-Thms.tex]{subst-subst}

Rather than include Agda proofs of all four lemmas, here we outline
the proof structure, analyse just one of the four proofs, and compare
fusion lemmas for λcl with the corresponding lemmas for λst.

A generic technique to prove fusion lemmas for STLC, including the
ones about renaming and substitution, is one of the main contributions
of ACMM \cite{DBLP:conf/cpp/Allais0MM17}. Their proof uses Kripke
logical relations and it relies on the invariant that corresponding
environment values are in appropriate relations, including when
environments are extended when going under a binder. Maintaining this
invariant is possible thanks to the generic framework for writing
traversals introduced by ACMM.

As it turns out, fusion lemmas for the closure language are simpler,
as they do not require the logical relation machinery of ACMM. This is
because renaming and substitution in λcl `do not go under binders`, as
can be seen from their definitions in
Section~\ref{sec:renam-subst}. For both renaming and substitution, in
the closure case (\AS{L}), the closure body is left untouched; only
the closure environment is modified.

We are now ready to take a closer look at the proof of the fusion
lemma stating that a renaming followed by a subsitution is a
substitution:

\ExecuteMetaData[StateOfTheArt/Closure-Thms.tex]{subst-rename} 
\ExecuteMetaData[StateOfTheArt/Closure-Thms.tex]{subst-rename-proof} 

The proof is by induction on the typing derivation of the term:

\begin{itemize}
\item In the variable case, the LHS and the RHS normalise to the same
  term, so \AS{refl} suffices.
\item In the application case, the proof is by induction and
  congruence.
\item In the closure case, the proof is also by congruence, but an
  equational proof is required to show that the LHS and RHS act in the
  same way on the environment \AS{E}.
\end{itemize}

The equational proof proceeds as follows:

\begin{enumerate}[nolistsep]
\item It uses the fact that function composition \AS{\_∘\_}
  distributes through mapping over environments \AS{\_<\$>\_}: we have
  \AS{f <\$> g <\$> E ≡ f ∘ g <\$> E} which is capture by the lemma
  \AS{< \$>-distr},
\item It uses the fact that when \AS{f} and \AS{g} are extensionally
  equal (\AS{∀ \{x\} → f x ≡ g x}), then \AS{f <\$> E ≡ g <\$> E} which
  is captured by the lemma \AS{<\$>-fun},
\item \AS{<\$>-fun} is instantiated with the inductive hypothesis.
\end{enumerate}

Unfortunately, Agda does not recognise this project's fusion lemmas as
terminating, and we were unable to provide a termination proof. Still,
we believe that the function does in fact terminate.


\chapter{Proving correctness of closure conversion with bisimulation}
\label{cha:prov-corr-clos}

Preceding sections defined the source and target languages of
closure conversion, λst and λcl, together with reduction rules for
each, and a closure conversion function \AS{min-cc} from λst to λcl.

The \AS{min-cc}  closure conversion is type- and
scope-preserving by construction. The property of type preservation
provides confidence in the compilation process, but in this
theoretical development which deals with a small, toy language, it is
within the reach of this project to prove properties about operational
correctness. 

One such operational correctness property of a pair of languages is
\textbf{bisimulation}. Intuition about bisimulation is captured by a
slogan: similar terms reduce to similar terms. 

This chapter starts by defining a relation between terms of λst and
terms of λcl, which we call a \textit{compatibility relation}. The
compatibility relation is syntactic: in general, two terms are
compatible when their subterms are compatible.

Then, we define what it means for a relation to be a bisimulation. A
bisimulation is a relation which has a semantic property which relates
reduction steps of source and target terms. Next, we will show that
the compatibility relation is a bisimulation.

Finally, we will link the compatibility relation to closure
conversion: we will argue that the graph relation of every sensible
closure conversion function is contained in the compatibility
relation. In particular, we will prove that this is the case for
\AS{min-cc}.

Overall, correctness of the minimising closure conversion is
established: first, by showing that the input and output of closure
conversion are related by a syntactic relation, and second, by showing
that this syntactic relation is also a semantic relation. Thus,
soundness of our closure conversion is established.

The part which shows that the compatibility relation is a bisimulation
is inspired by the `Bisimulation' chapter from
\cite{DBLP:conf/sbmf/Wadler18}. 

\subsection{Compatibility relation}
\label{sec:comp-rel}

The compatibility relation is defined as follows:

\begin{definition}
  Given a term \AS{M} in λst and a term \AS{M†} in λcl,
  the compatibilty relation \AS{M \tis M†} is defined inductively as
  follows:

  \begin{itemize}
  \item (\textit{Variable}) For any given variable (proof of context
    membership) \AS{x}, we have \AS{S.` x \tis T.` x}.

  \item (\textit{Application}) If \AS{M \tis M†} and \AS{N \tis N†},
    then \AS{M · N \tis M† · N†}.

  \item (\textit{Abstraction}) If \AS{N ~ T.subst (T.exts E) N†}, then
    \AS{S.L N \tis T.L N† E}.
    
  \end{itemize}
\end{definition}

Recall that λst and λcl share types, contexts, and variables (proofs
of context membership). In fact, compatibility is only defined for source
and target terms of the same type in the same context (this is
explicit in the Agda definition).

While the variable and application cases are straightforward, the
abstraction / closure case needs some explanation. Since the body
\AS{N} of the abstraction is defined in \AS{σ ∷ Γ}, and the body of
the closure \AS{N†} is defined in \AS{σ ∷ Δ}, they cannot be
compatible. However, \AS{N} can be compatible with the result of
substituting the environment \AS{E} in \AS{N†} (the environment is
extended with a variable corresponding to the \AS{σ} in the
context). The intuition for the abstraction/closure case is that
substituting the environment `undoes' the effect of closure conversion
on the context. 

The compatibility relation is defined in Agda as follows:

\ExecuteMetaData[StateOfTheArt/Bisimulation.tex]{tilde}

We have defined the syntactic compatibility relation. The next
section defines what it means for a relation to be a bisimulation.

\section{Bisimulation}

Bisimulation, as the name implies, is defined in terms on two
simulations: one from source to target terms, and the other one from
target to source terms.

In the following definitions speak about a two languages, \AS{A} and
\AS{B}. Also, whenever simulations or bisimulations are mentioned,
they are implicitly \textit{lock-step}. The literature has example of
more general simulations.

\begin{definition}
  Given a relation \AS{≈} between terms of \AS{A} and terms of \AS{B},
  we say that \AS{≈} is a \textbf{simulation} from \AS{A} to \AS{B} if
  and only if for all terms \AS{M} and \AS{N} in \AS{A}, and \AS{M†}
  in \AS{B}, if \AS{M} reduces in a single step to \AS{N}, then there
  exists a term \AS{N†} in \AS{B} such that \AS{M†} reduces to \AS{N†}
  in a single step, and \AS{N} is in the \AS{≈} relation with \AS{N†}:
  \AS{N ≈ N†}.
\end{definition}

The essence of simulation can be captured in a diagram.

\[ \begin{tikzcd}
M \arrow{r}{\longrightarrow} \arrow[swap]{d}{\approx} & N \arrow{d}{\approx} \\%
M \dagger \arrow{r}{\longrightarrow}& N \dagger
\end{tikzcd}
\]

Recall that the \textit{converse} of the relation \AS{≈} is a relation
\AS{≈'} defined by \AS{y ≈' x} whenever \AS{x ≈ y}.

\begin{definition}
  A relation \AS{≈} is a \textbf{bisimulation} if and only if it is a
  simulation and its converse is also a simulation.
\end{definition}

In Agda, we instantiate the definition of simulation twice: once for a
simulation from λst to λcl, and again for a simulation from λcl to
λst:

\ExecuteMetaData[StateOfTheArt/Bisimulation.tex]{simulation}

Then we can provide an Agda definition of a bisimulation:

\ExecuteMetaData[StateOfTheArt/Bisimulation.tex]{bisimulation}

To show that the compatibility relation is a bisimulation, we need to
obtain lemmas about the interactions between the compatibility
relation, values, renaming, and substitution.

\section{Compatibility, values, renaming, and substitution}
\label{sec:comp-valu-renam}

As discussed in [TODO], mechanising the meta-theory of a language
involves proving lemmas about the interactions between various
traversals and transformations, including renaming, substitution, and
compilation phases. This is also the case for proving correctness with
bisimulation, which requires establishing lemmas about the interplay
between the compatibility relation, values, renaming, and
substitution. In fact, proving those lemmas often constitutes the
biggest effort in the entire proof. In Chapter~\ref{cha:refl-eval}, we
reflect on the possibiity of automating this effort with generic
proving.

For each relevant property, we state it as an informal lemma, give its
Agda statement, and its Agda proof.

\begin{lemma}{Values commute with compatibility.}
  If \AS{M \tis M†} and \AS{M} is a value, then \AS{M†} is also a
  value.
\end{lemma}

The proof is by cases of term constructors.

\ExecuteMetaData[StateOfTheArt/Bisimulation.tex]{val-comm}

\begin{lemma}{Renaming commutes with compatibility.}
  If \AS{ρ} is a renaming from \AS{Γ} to \AS{Δ}, and \AS{M \tis M†} are
  compatible terms in the context \AS{Γ}, then the results of renaming
  \AS{M} and \AS{M†} with \AS{ρ} are also compatible: \AS{S.rename ρ M
    \tis T.rename ρ M†}.
\end{lemma}

The proof is by induction on the similarity relation.

\ExecuteMetaData[StateOfTheArt/Bisimulation.tex]{rename-comm} 

The variable and application cases are straightforward, but as ever, the
abstraction case is more involved: it requires rewriting with an
instantiation of the fusion lemma \AS{rename∘subst}.

\ExecuteMetaData[StateOfTheArt/Closure-Thms.tex]{lemma-ren-comm}

The final lemma is about the interplay between compatibility and
substitution. 

\begin{definition}
  Suppose \AS{ρ} and \AS{ρ†} are two substitutions which take
  variables \AS{x} in \AS{Γ} to terms in \AS{Δ}, such that for all
  \AS{x} we have that \AS{lookup ρ x \tis lookup ρ† x}. Then we say
  that \AS{ρ} and \AS{ρ†} are \textit{pointwise compatible}.
\end{definition}

\begin{lemma}
  \textit{Substitution commutes with compatibility}. Suppose \AS{ρ}
  and \AS{ρ†} are two pointwise compatible substitutions. Then given
  compatible terms \AS{M \ti M†} in \AS{Γ}, the results of applying
  \AS{ρ} to \AS{M} and \AS{ρ†} to \AS{M†} are also compatible:
  \AS{S.subst ρ M \ti T.subst ρ† M†}.
\end{lemma}

Pointwise similarity relation between substitutions \AS{ρ} and \AS{ρ†}
is defined in Agda with \AS{\ti σ}:

\ExecuteMetaData[StateOfTheArt/Bisimulation.tex]{pointwise-sim}

We can show that pointwise similarity is preserved by
applying \AS{exts} to both substitutions:

\ExecuteMetaData[StateOfTheArt/Bisimulation.tex]{pointwise-sim-exts}

In fact, exteding pointwise-similar substitutions with similar terms 
preserves pointwise similarity:

\ExecuteMetaData[StateOfTheArt/Bisimulation.tex]{pointwise-sim-extend}

With the notion of pointwise similarity, we can prove that
substitution commutes with similarity:

\ExecuteMetaData[StateOfTheArt/Bisimulation.tex]{subst-comm}

Just like in the lemma that renaming commutes with compatibility, the
only non-trivial case is the one about abstractions/closures, which
requires rewriting by an instatiation of the fusion lemma
\AS{subst∘subst}.

\ExecuteMetaData[StateOfTheArt/Closure-Thms.tex]{lemma-subst-comm}

With those three lemmas, showing that the compatibility relation is a
bisimulation becomes straightforward.

\section{Compatibility relation is a bisimulation}
\label{sec:comp-relat-bisim}

The proof that the compatibility relation \AS{\ti} is a bisimulation
consists of two proofs of simulations: \AS{st-sim : ST-Simulation
  \_\ti\_} and \AS{ts-sim : TS-Simulation \_\ti\_}. Taken together,
they prove that \AS{\ti} is a bisimulation:

\ExecuteMetaData[StateOfTheArt/Bisimulation.tex]{bisim}

[TODO analyse the proof in detail or refer the reader to the technical
appendix and to PLFA?]

\section{Compatibility relation and closure conversion}
\label{sec:comp-relat-clos}

We have showed that the compability relation is a bisimulation. The
connection between closure conversion and the compatibility relation
is that we require that the graph relation of every-well behaved
closure conversion function \AS{\_†} is contained in the compatibility
relation: \AS{M \tis M †}. We cannot quantify over all closure conversion function, so
instead, we must show that this is the case for every function which
we claim is a well-behaved closure conversion. In this section, we
will show that this property is possessed by the trivial closure
conversion \AS{simple-cc} and the minimising closure conversion
\AS{\_†}.

While there are many possible closure conversion function, which
differ by how big environments they construct, there is a unique
backtranslation from λcl to λst, which we call \AS{undo}. We can show
that the converse of the graph relation of \AS{undo} is contained in
the compatibility relation: \AS{undo N \tis N}.

\ExecuteMetaData[StateOfTheArt/ClosureConversion.tex]{undo-compat}

\noindent\fbox{%
    \parbox{\textwidth}{%
      \paragraph{Side note about proving termination of proofs.}
      Notice that several functions in this development were annotated
      as TERMINATING. This annotation is not checked, and if a
      function is annotated incorrectly, it could cause Agda to loop
      forever during typechecking. In general, a function terminates
      if it strictly decreases in one of its arguments, and the type
      of that argument cannot decrease infinitely: e.g. natural
      numbers are bounded from below by zero.

      Agda can tell that an argument decreases when it is evident
      syntactically, but in more complicated cases, an explicit proof
      needs to be provided. For the \AS{undo} function, a sketch of
      such proof is as follows: define a \AS{size} measure (function)
      on target terms which is inductively defined as the number of
      term constructors in the term, including in the
      environment. Then, in the closure case of \AS{undo}, we can
      argue that \AS{T.subst (T.exts E) M} is smaller than \AS{ T.L M
        E}.

      A technique which formalises such arguments is called
      \textit{well-founded induction}. We do not formalise this
      argument for the \AS{undo} function, as this would involve
      arithmetic reasoning which is tedious in Agda.
      
    }%
  }

The trivial closure conversion \AS{simple-cc} uses full contexts as
environments (through identity substitutions \AS{id-subst}):

\ExecuteMetaData[StateOfTheArt/Bisimulation.tex]{convert}

\AS{simple-cc} is well-behaved as its graph relation is contained in
the compatibility relation. The proof is by straightforward induction;
in the abstraction case, we need to argue that applying an identity
substitution leaves the argument term unchanged.

\ExecuteMetaData[StateOfTheArt/Bisimulation.tex]{graph}

The minimising closure conversion \AS{\_†} is also well-behaved:

\ExecuteMetaData[StateOfTheArt/ClosureConversion.tex]{min-cc-sim-t}

The proof of this is too long to discuss here, but the reader can find
it in the technical appendix of this report.

***

This concludes the argument that our closure conversion is correct. We
have shown that the graph relation of our closure conversion function
is contained in the compatibility relation, and that the compatibility
relation is a bisimulation. This means that when a source term and its
closure converted target term both take a reduction step, then the
terms they reduce to are also compatible.

[TODO example that reductions may take us out of the graph relation of
a specific closure conversion function]

\chapter{Proof by logical relations}

The previous chapter showed a correctness property of closure
conversion: the source and target of our closure conversion are in a
relation which is a bisimulation. This chapter demonstrates another
technique for showing correctness properties: Kripke-style [??]
type-indexed logical relations. Type-indexed logical relations are
characterised by using induction on the type structure of terms.

The outline of this chapter is similar to that of the previous one
about bisimulations. First, we introduce a modified representations of
simply typed lambda calculus (λst') and the language with closures
(λcl'), where terms are explicitly labelled as values or reducible
expressions (this helps with mechanisation as logical relations treat
values and reducible terms differently). Unlike λst and λcl, λst' and
λcl' semantics are defined as big step [TODO wording].

Then, the syntactic compatibility relation between λst' and λcl' is
redefined. Finally, we define a logical relation between λst' and
λcl', and we formulate the fundamental theorem for the that logical
relation. As a corollary, it follows that the compatibility relation
implies the logical relation for closed terms.

The proof by logical relations is based on
\cite{DBLP:conf/popl/MinamideMH96}, but the Agda mechanisation is this
project's contribution.

\section{Alternative representation of languages}
\label{sec:altern-form-interm}

This section presents an alternative representation of the source and
target languages of closure conversion. We call the new formalisation
of the source language λst', and the new formalisation of the target
language - λcl'. Compared with λst and λcl in
Chapter~\ref{cha:agda-development}, λst' and λcl' are different in two
ways. Firstly, the distinction between values and non-values is made
explicit in the definition of terms in λst' and λcl', replacing a
predicate on terms in λst and λcl. Secondly, we give big-step
semantics for λst' and λcl', in contrast to small-step semantics for
λst and λcl. These two differences simplify mechanisation of a proof
by logica relation.

These improvements in formalisation are inspired by an Agda
formalisation accompanying \cite{DBLP:conf/cpp/McLaughlinMS18}.

[TODO maybe only discuss STLC?]

The definitions of types, contexts, variables as proofs of context
membership, and environments, are the same as for λst and λcl in the
previous chapter. The definition of language expressions is different,
however, in that it makes an explicit distinction between values \AS{Val} and
non-values \AS{Trm}. This is achieved by indexing the \AS{Exp} data
type by a \AS{Kind}:

\ExecuteMetaData[LR/Base.tex]{kind}

\ExecuteMetaData[LR/STLC.tex]{exp}

Notice that there are two new constructors for language
expressions. The first one is \AS{`val}, which takes a value \AS{Val}
to a term \AS{Trm} and thus makes it possible to use values in
positions where terms are expected. The second is \AS{`let}, which is
a standard let construct. The let is necessary to make the evaluation
order explicit: function application applies a value to a value, so
nested computations need to be factored out and bound as values by a
let expression. This representation is known as A-normal form
\cite{DBLP:conf/lfp/SabryF92}; its other benefit is that it simplifies
the definition as big-step semantics.

Definition of renaming and substitution are similar to those for
λst, so we do not include the updated versions here.

We define aliases for closed values \AS{Val₀} and closed terms
\AS{Trm₀} (typed in an empty context):

\ExecuteMetaData[LR/STLC.tex]{ground}

Like it was mentioned, the semantics of λst are defined as big-step
semantics [TODO wording]. Given a term \AS{M} and a value \AS{V}, the
inductive definition \AS{M ⇓ V} states the conditions for \AS{M} to
reduce to a value \AS{V}:

\ExecuteMetaData[LR/STLC.tex]{big-step}

It is worth explaining the \AS{⇓step} constructor and the \AS{M →₁ M'}
data type. The \AS{M →₁ M'} data type describes part of the small-step
reducton relation and has a single constructor which captures beta
reduction for functions. The \AS{⇓step} constructor is similar to the
transitive closure of the small-step reduction relation: if \AS{M}
reduces to \AS{M'} in a single step, and \AS{M'} reduces to \AS{V} in
multiple steps, then \AS{M} reduces to \AS{V} in multiple steps.

Finally, ... [TODO what to make of a non-terminating proof of
termination?]

\ExecuteMetaData[LR/STLC.tex]{sn}

Differences between λcl and λcl' are analogous.

\section{Correctness by logical relations}
\label{sec:corr-logic-relat}

This section defines two relations between terms of λst' and λcl'. The
first is just a reformulation of the compatibility relation from
Chapter~\ref{cha:prov-corr-clos}, which, as the reader may remember,
subsumes the graph relation of any closure conversion. The other is a
\textit{logical relation}, which captures the notion that related
terms reduce to related values.

We set up a fundamental theorem for the logical relation, and as a
corollary, we obtain the result that for closed terms, the compability
relation implies the logical relation.

\begin{lemma}
  \label{sec:corr-logic-relat-1}
  \emph{Correctness property for the compatibility relation:} Given a
  closed source term \AS{M}, and a closed target term \AS{M†}, if
  \AS{M} and \AS{M†} are compatible, then they reduce to values which
  are in the logical relation.
\end{lemma}

From this, a correctness property for closure conversion follows:
given any well-behaved closure conversion function \AS{\_†}, a closed
source term \AS{M}, and a closed target term \AS{M†} which is the
result of closure converting \AS{M}, since \AS{M} and \AS{M†} are
compatible, they reduce to related values.

The proof is inspired by a sketch of an argument from
\cite{DBLP:conf/popl/MinamideMH96}.

\subsection{The compatibility relation}
\label{sec:lr-compat-rel}

We denote the compatibility relation with \AS{≅}.  In general, given a
term \AS{M₁} in λst' and a term \AS{M₂} in λcl', \AS{M₁} and \AS{M₂}
are compatible (M₁ ≅ M₂) when their subterms are compatible. In the
special case of abstractions/closures, the closure body is renamed
with the environment in the premise of the rule.

\ExecuteMetaData[LR/LR.tex]{compat}

For brevity, we do not include translation functions from λst' to
λcl'. The reader should convince themselves that the minimising
closure conversion from λst and λcl could be ported to λst' and λcl',
and that its graph relation woud be contained in \AS{≅}.

\subsection{The logical relation}
\label{sec:logical-relation}

While the compatibility relation captures syntactic correspondence, we
need another relation on (closed) language expressions which captures
operational correspondence. We define a family of logical relation
\AS{⇔} relating closed source terms (reducible expressions) to closed
target terms (\AS{\ti}) and closed source values to closed target
values (\AS{≈}).  The relations are defined by induction on types. In
the definition, we write \AS{τ ∋ M₁ ~ M₂} or \AS{τ ∋ M₁ ≈ M₂} to mean
that \AS{M₁} and \AS{M₂} are related at type \AS{τ}:

\begin{tabular}{rccl}
  \AS{τ ∋} & \AS{M₁ \tis M₂}   & iff  & \AS{M₁ ⇓ V₁}, \AS{M₂ ⇓ V₂}, and \AS{τ ∋ V₁ ≈
                                 V₂}  \\
  \AS{σ ⇒ τ ∋} & \AS{U₁ ≈ U₂} & iff & for all \AS{σ ∋ V₁ ≈ V₂}, \AS{τ ∋ U₁
                               `\$ V₂ \tis U₂ `\$ V₂ }
\end{tabular}


There is no case for \AS{≈} at the ground type \AS{α} as only
variables can have the ground type, and the values in \AS{≈} are
closed.

In Agda, \AS{\ti} and \AS{≈} are defined as specialisations of the
\AS{⇔} relation on closed expressions of λst and λcl.

\ExecuteMetaData[LR/LR.tex]{related}

We define a pointwise version of the \AS{≈} relation which
relates source and target substitution environments, similar to what
we did in Section~\ref{sec:comp-valu-renam}:

\ExecuteMetaData[LR/LR.tex]{pointwise-related}

We also provide a function \AS{∙\tss{R}} which extends two related
substitution environments with a pair of related values:

\ExecuteMetaData[LR/LR.tex]{pointwise-ext}

Finally, we can state the fundamental theorem for our logical
relation [TODO wording]. The theorem generalised the Correctness
property for the compatibility relation to open terms, as long as
there are substitution environments in the pointwise relation.

\begin{lemma}
  \emph{Fundamental theorem of logical relations.} Given a source
  term \AS{M}, a target term \AS{M†}, a source substitution
  \AS{ρ\tss{s}}, and a target substitution \AS{ρ\tss{t}}, if \AS{M} is
  compatible with \AS{M†}, and for all variables \AS{x} in the
  context, the corresponding values in the substitution environments
  are in the logical relation (\AS{ρ\tss{s}(x) ≈ ρ\tss{t}(x)}), then
  \AS{S.subst ρ\tss{s} M} and \AS{T.subst ρ\tss{t} M†} are in the
  logical relation.
\end{lemma}

\ExecuteMetaData[LR/LR.tex]{fund-t}

Observe that the Fundamental theorem, instatiated from closed terms,
is equivalent to the Correctness property for the compatibility relation.

We do not include the Agda proof here as it is not very readable;
instead, we present several cases on paper: TODO

[TODO let case in the proof]

\chapter{Reflections and evaluation}
\label{cha:refl-eval}

This project is a case study on verification of transformations of
functional programs using two different techniques: bisimulations and
logical relations. The implemented transformation is closure
conversion. Both proofs of operational correctness are mechanised with
state-of-the-art techniques.

The \textbf{original objectives} of the project were:

\begin{enumerate}
\item To implement a compiler transformation for a variant of
  simply-typed lambda calculus in Agda.
\item To use scope-safe and well-typed representation for the object
  languages.
\item To prove that the transformation is correct: that the output
  program of the transformation behaves `the same` as the input
  program.
\item To use generic programming techniques from ACMM.
\end{enumerate}

The \textbf{contributions} of this project are as follows:

\begin{enumerate}
\item All the original objectives were achieved.
\item It is demonstrated that languagues with closures and closure
  conversion are problematic for current state-of-the-art techniques
  for mechanising language meta-theory.
\end{enumerate}

\paragraph{(Objective 1) Capturing the essence of closure conversion}
The implemented tranformation --- closure conversion --- requires a
different source and target language. While the formalisation of the
source language is largely borrowed from ACMM, and the formalisation
of the target language is similar except for the difference between
abstractions and closures, this project's contribution was to capture
the essence of closure conversion in what we believe is the simplest
and most elegant way possible.

\paragraph{Inherently typed closures}
A traditional representation of closure conversion replaces variables
in the source program with references to a record containing the
environment in the target program. This project's use of scope-safe
and well-typed terms allowed for a more elegant solution where the
closure body is typed in a context corresponding to the closure's
environment, and variables remain variables.

\paragraph{Closure environments as substitution environments}
Furthermore, while a closure environment is traditionally represented
as a record which stores environment values, this project captures the
essence of an environment by representing it as a substitution
environment, i.e. a mapping from variables to values.

\paragraph{Existential types for closure environments}
As this report points out, closure environment must have existential
types in order for a program with closures to be well-typed. This
observation was made by \cite{DBLP:conf/popl/MinamideMH96}, which
deals with this fact by equipping the closure language with
existential types. This project uses a different, arguably simpler
approach, whereby closure environment are existentially typed
\textit{in the meta language (Agda)}, which allows us to keep object
langauge types simple.

\paragraph{Comparison with traditional closure conversion}
In comparison with traditional closure conversion which represents
environments as records, this formulation, which represents closure
environments as substitution environments, i.e. meta-language
functions, is further removed from the eventual target, which is
machine code. But one can imagine a subsequent compilation phase which
replaces substitution environments with records, and variables with
record lookups (the object language would need existential types
then). In general, splitting the compilation process into many
specialised passes facilitates verification, as each compilation phase
is easier to verify, and composing correctness results about phases
gives rise to a end-to-end correctness result.

\paragraph{(Objetive 2) Scope-safe and well-typed representation}
Both the source and target language have scope-safe and well-typed
representation, which were possible thanks to using dependently-typed
Agda as the meta language. Using inherently scoped and typed terms has
many benefits, which include the fact that when programs are synonymous
with their typing derivations, transformations on programs are
synonymous with proofs of type preservation. Additionally, many
techniques for reasoning about operational correctness are type
directed, e.g. the type-indexed logical relations which we used, and
inherently typed representations are well-matched to such techniques.

\paragraph{(Objetive 3) The closure conversion preserves operational correctness}
This project uses two standard techniques to show that the implemented
closure conversion is correct: bisimulation and logical relations. In
an informal setting of pen-and-paper proofs, both of those techniques
have rather straighforward proofs. However, mechanisation of those
proofs involves proving several lemmas about the interactions between
renaming, substitution, closure conversion, and the compatibility
relation.

\paragraph{Mechanising the meta-theory of a language}
As observed in ACMM, mechanising the meta-theory of a language most
often requires proving lemmas about the interactions between different
transformations, or semantics, like renaming and substitutions. ACMM
singles out synchronisation lemmas, which relate two semantics
(e.g. for every renaming there exists a substitution which behaves the
same), and fusion lemmas (e.g. for every composition of two
substitutions, there exists a substitution which behaves the
same). 

\paragraph{(Objetive 4) ACMM}
ACMM exploit similarity between various traversals (semantics) on
simply typed lambda calculus (STLC) to come up with a generic way to
prove synchronisation and fusion lemmas for STLC. Indeed, this project
uses fusion lemmas about STLC from ACMM in the proof with logical
relations.

Intermediate languages other than STLC require their own definitions
of renaming and substitution, and proofs of correctness lemmas. For
example, the proofs of operational correctness with bisimulations and
logical relations depend on four fusion lemmas relating renaming and
substitution for the language with closures. In fact, proving those
correctness lemmas was the biggest effort in the whole proof.

\paragraph{Possible remedy: AACMM and generic programming}
The problem of having to define renaming and substitution for each new
language, and proving correctness lemmas about the interactions
between renaming, substitution, and transformations, is address by a
follow-up paper, which we will refer to as AACMM
\cite{DBLP:journals/pacmpl/AllaisA0MM18}. AACMM provides a way to
supply a definition of a syntax with bindings, and then derives
meta-theoretical correctness lemmas from that definition. The paper
repository contains an example of an elaboration whose source is a
language with a let construct, and whose target is simply-typed lambda
calculus with let-expressions inlined. The example demonstrates how a
proof of simulation is drastically simplified thanks to the use of the
AACMM library and its generic programing capabilities.

\paragraph{Feasibility of closure conversion in AACMM}
AACMM demonstrates that transformations like let-inlining and CPS
conversion can be expresses in their generic framework. They pose an
open question about which compilation passes can be implemented
generically. Unfortunately, this work suggests that closure conversion
might not fit well into the AACMM framework. Specifically, the
closure language in this project --- with aforementioned features like
syntax being mutually dependent on substitution environments, or
environments being existentially quatified in the meta language (Agda)
--- is not expressible as an AACMM generic syntax. The traditional
representation of a languages with closures --- with environments as
records and existential types in the object language --- would not fit
either as syntaxes in AACMM cannot contain existential types.

\paragraph{Bisimulation vs logical relations}
[TODO what could I say about the pros and cons of both?]
[Derek Dreyer's paper]

\paragraph{Summary}
This work and ACMM/AACMM are both concerned with mechanising the
meta-theory of languages, and applying this metatheory to reason about
the language. While AACMM shows that a certain class of languages /
syntaxes can be treated generically, this work contains a negative
result which indicates that a language with closures might not benefit
from current techniques for relieving the burden of mechanising
meta-theory. This is an open question, however, whether there exist
feasible generic syntaxes which would encompass a language with
closures, or whether an alternative formalisation of a language with
closures exists which is compatible with AACMM.

[TODO the conclusions of this chapter depend on my understanding of
AACMM --- is it correct?]

\chapter{Relationship to the UG4 project}

The UG4 project and this work explore two ends of a spectrum of
transformations on programs. While this work deals with compilation
phases, the UG4 project was concerned with program derivations. This
chapter attempts to find common characteristics between compilation
phases and program derivations, but also highlight their
differences. It also looks at the middle of the spectrum, where
certain transformations on programs could be classified either way. It
ends with a reflections on lessons learned from this year's project
which would have been helpful for last year's work.

In literature, what we refer to as program derivation is sometimes
called program construction or derivation, and a program derivation
described an instance of the process. Here, we use the term
`derivation' to describe a family of transformations on programs, in
order to avoid confusion with program transformations within compilers.

\section{Compilation phases and program derivations}
\label{sec:comp-phas-progr}

Compilation phases and derivations can be described and compared in
terms of several criteria: (a) the objective of the transformation,
(b) what the source and the target of the transformation is, (b)
required expertise from the user, (c) user's expertise and input
needed to guide the transformation, and (d) whether rules apply in all or
only selected possible places.

The following characterises compilation phases:

\begin{enumerate}
\item (\textbf{Objective} and \textbf{Source and target of
    transformation}) The objective is to generate low-level code from
  a program in the source language.
\item (\textbf{Required expertise}) Programmer only needs to know the
  source language
\item (\textbf{Input from programmer and required expertise}) Little or none, except for
  specialised annotations which are used by the compiler
\item (\textbf{Totality and selectivity}) If a compilation phases can
  be specified as a rule, then the rule is typically applied in all
  the places where its premises match
\item (\textbf{Examples}) Closure conversion, CPS transformation,
  lambda lifting, type-checking, constant expression folding, code
  generation, dead code elimination, inlining.
\end{enumerate}

The following are features of program derivation:

\begin{enumerate}
\item (\textbf{Objective}) Enable the programmer to write clear, concise,
  understandable programs which serve as specification. Methodically
  derive a correct, efficient implementation. Possibly mechanise the
  tranformation described by the derivation.
\item (\textbf{Source and target of transformation}) The source could be an
  executable functional program or a non-executable specification
  (e.g. in the categorical calculus of relations; or as a solution to
  an equation). The target is an efficient functional or imperative
  program. 
\item (\textbf{Input from programmer and required expertise}) A derivation is a
  description of a transformation which can be calculated
  mechanically, so by definition, non-trivial derivations require
  input fromt the programmer. This might require expertise beyond the
  capabilities of an average programmer.
\item (\textbf{Totality and selectivity}) Rules are applied selectively: in arbitrary places and order.
\item (\textbf{Example realisation}) Bird-Meertens formalism: equational
  reasoning, where rules justify correctness of each step \cite{gibbons1994introduction}.
\item (\textbf{Obstacles to adoption}) Few tools, hard to learn, hard to use,
  hard to understand, hard to maintain, writing the implementation by
  hand can be easier than writing the derivation.
\end{enumerate}

Benefits of employing a sort of program derivation (more or less
formal) for the programmer include: (a) a structured process of
obtaining an implementation from specification, (b) greater confidence
in the correctness of the implementation, (c) possibility of
discovering further optimisations, and finally (d) a framework for a
proof of correctness. This last use case could be explained as
follows: suppose we can prove correctness of the "specification"
program, and that each step preserves the meaning of the program. Then
we can show correctness of the "implementation" program.

There are many flavours of program derivation, but in subsequent
sections, we will assume a particular style known as Bird-Meertens
Formalism \cite{gibbons1994introduction}. In BME, derivations are
based on equational reasoning. A derivation is a sequence of
intermediate forms of the program, where the steps are annotated with
rules justifying the step. This is illustrated by the template below:

\begin{verbatim}
specification
  = { justification }
intermediate form 1
  = { justification }
...
  = { justification }
intermediate form n
  = { justification }
implementation
\end{verbatim}

\section{Rewrite rules}
\label{sec:rewrite-rules}

When equational reasoning is employed, derivation steps can be
justified with \textit{rewrite rules}. A basic example of a rewrite
rule in functional programming is \textit{map-comp}:

\AS{∀ \{f g xs\} → map f (map g xs) ≡ map (f ∘ g) xs}

where $f$, $g$, $xs$ are metavariables. An application of a rewrite
rule consists of unifying the LHS of the rule with a subterm of the
program (and thus obtaining a substitution σ), and then replacing the
subterm with the RHS of the rule, instantiated with the substitution
σ.

TODO fix Notably, such simple form of a rewrite rule only supports first-order
abstract syntax trees, but not higher-order abstract syntax. (TODO
elaborate on what it would mean to have a context and go under a
binder in a rewrite rule). (TODO can't express conditions) 

\section{Program derivations in compilers and their limitations}
\label{sec:progr-deriv-comp}

There are program transformations in existing compilers which, other
than being compilation phases, have features of program derivation.  A
good example of this is the support for rewrite rules in GHC, a
Haskell compiler. A Haskell programmer may specify a rule like
\textit{map-comp} as part of the code, and in one of early compilation
passes, GHC will apply the rule wherever possible (i.e. replace the
occurrences of the LHS with the RHS). Rewriting in GHC is a
compilation phase in the sense that rules are applied in all places
they match, but it also resembles derivations since the transformation
is guided by input from the programmer, namely, the specified rewrite
rules. Note that GHC makes no attempt to ensure that the rules
preserves the meaning, or that rewriting would terminate: a programmer
could externally check these properties, e.g. using a proof
assistant.

Yet another ambiguous situation is where program transformation
becomes a search problem. A compiler could try to find a
transformation by applying rewrite rules in a selective,
non-deterministic manner, and thus perform a search over the space of
possible derivations. The search could be guided by an objective
function, for example, a sort of static analysis of the running time,
or the compiler could evaluate programs by running them on sample
inputs. Exploring the space of derivations is the approach taken by
the Lift compiler \cite{TODO}.

Programmer's input may or may not be involved in such search. For
example, the UG4 project delivered a graphical user interface which
allows the user to interact with the derivation process.

Unfortunately, derivation search needs a fixed set of rewrite rules,
either hardcoded in the compiler or provided by the user. This
precudes derivations which use original rewrite rules, coming up with
which would require human insight. Section~\ref{sec:progr-deriv-agda},
which discusses some of the rewrite steps from the UG4 project, has
examples of rules which were invented for a specific derivation, and
it is difficult to conceive that they could all be provided to the
compiler in advance.

\section{Program derivations in the UG4 project}
\label{sec:progr-deriv-ug4}

The UG4 project analyses two instances of program derivation in
detail. The first one is a derivation, described in
\cite{gibbons1994introduction}, of an efficient implementation of the
maximum segment sum problem (MSS). While the input specification
(which is also a runnable program) runs in cubic time in the length of
the input list, the output program runs in linear time. The asymptotic
speed-up is achieved by applying several rewrite rules involving
higher-order functions on lists such as map, foldr, and filter.

The second case study involved an original derivation of a program for
matrix-vector multiplication. The input program takes a dense matrix,
and the output program takes a sparse matrix in the compressed sparse
row (CSR) format. Or, to be precise, the input program which acts on a
dense matrix, is transformed into a composition of two programs: (a) a
conversion from a dense to a CSR-sparse matrix, and (b) a
matrix-vector multiplication program which acts on a CSR-sparse
matrix. This is because, as a rule, the input and output types of the
program must stay the same in the course of the derivation. This
second derivation was similarly accomplished with rewrite rules
involving higher-order functions.

\section{Implementation of rewriting in the UG4 project}
\label{sec:impl-rewr-ug4}

The UG4 project included a purpose-built framework for specifying
derivations. The framework included:

\begin{enumerate}
\item A simple functional language with parametric polymorphism. The
  language is point-free, that is, based on function composition
  rather than variable binders. This is because variables and
  abstraction are difficult to implement correctly, as demonstrated by
  this UG5 project, and even more difficult to rewrite.

\item A type-checker for the language.

\item Rewriting functionality and declaring derivations as sequences of
  rewrites.

\item An interpreter for the language, which was used to empirically
  verify claims about performance gains from derivations.
\end{enumerate}

Writing the framework was a good exercise in implementing routine
parts of compiler front-ends, such as type checking and
unification. Writing it in Scala made sense given the stretch
objective of compiling the language to Lift, which was not realised,
however.

Rewrite rules were stated without justification, much as postulates in
Agda. One could prove the rules externally – but then one is pressed
to ask, why not express a derivation in a proof assistant, which
supports unification and rewriting natively? Indeed, with hindsight,
we can say with certainty that a proof assistant is perfectly suited
for the job, its only downside being that it requires considerable
expertise, which I did not have during my fourth year. [TODO I/me?]
The next section expresses a part of the derivation from the UG4
project in Agda, and evaluates the benefits of doing so, compared with
the approach from last year's project.

\section{Program derivation in Agda: sparse matrix-vector
  multiplication}
\label{sec:progr-deriv-agda}

To complete last year's work, we conduct a derivation of the program
for matrix-vector multiplication which acts on CSR-sparse
matrices. Unlike last year, we can now provide proofs of individual
rewrite rules. Indeed, some proofs are quite involved. TODO whether
and how to do it.

% use the following and \cite{} as above if you use BibTeX
% otherwise generate bibtem entries
\bibliographystyle{plain}
\bibliography{main}

\chapter{Conclusion}
\label{cha:conclusion}

The main deliverables of this project are (a) an elegant
representation of a language with closures, (b) a type-preserving
closure conversion algorithm which minimises closure environments, and
(c) mechanisations of proofs of correctness of closure conversion
using two different techniques: bisimulations and logical relations.

This work builds on a long line of research in several areas:
higher-order abstract syntax representation, type-preserving
compilation, and compiler verification. In particular, the style of
the Agda development is influenced by ACMM and PLFA.

By mechanising two different proofs of correctness of the
transfomation, the project provides a reference for comparing the
methods of bisimulation and logical relations.

It was confirmed that when the languages have a type- and scope-safe
representation, a large amount of effort in mechanising
meta-theoretical proofs goes into correctness lemmas about
interactions between renaming, substitution, and other traversals.

TBC



\chapter{Technical appendix}
\label{cha:technical-appendix}

\section{Minimising closure conversion and the compatibility relation}
\label{sec:minim-clos-conv-2}

Below is the Agda proof that the graph relation of the minimising
closure conversion function is contained in the compatibility relation.

\ExecuteMetaData[StateOfTheArt/ClosureConversion.tex]{min-cc-sim}

\section{Compatibility relation is a bisimulation}
\label{sec:comp-relat-bisim-1}



\end{document}
